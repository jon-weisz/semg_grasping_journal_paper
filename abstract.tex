This thesis describes contributions towards the implementation of Human-in-the-Loop (HitL) grasping for assistive robotics, with a particular focus on low throughput, high noise interfaces such as electroencephalography (EEG) or electromyography (EMG) brain-computer interface(BCI) devices in natural environments. Although progress in the robotics field has been swift, it is unlikely that truly independent operation of robots in situations where they will interact closely with objects, obstacles, and perhaps even other people in their environment will evolve in the immediate future. However, with the help of a human operator, it is possible to achieve robust, safe operation in complex environments. This work describes a system that can accomplish this with minimal interfaces that are accessible even to individuals with impairments, which will enable the development of more capable assistive devices for these individuals. 

Grasping an object generally requires contextual knowledge of the object and the intent of the user. We have developed a user interface for an on-line grasp planner that allows the user to effectively express their intent. Grasping in natural environments requires grasp planning algorithms that are robust to target localization errors.  This work describes grasp quality measurements that generate more robust grasps by considering the local geometry of the object as well as how uncertainty will affect the proposed grasp. These new measures are integrated into an augmented reality interface that allows a user to plan a grasp online that matches their intent for using the object that is to be grasped.  This interface is validated by testing with real users, both healthy and impaired, using a variety of input devices suitable for impaired subjects, such as low cost EEG and EMG devices. This work forms the foundation for a flexible, fully featured HitL system that will allow users to grasp objects in cluttered spaces using novel, practical BCI devices that have the potential to bring HiTL assistive devices out of the research environment and into the lives of those that need them. 
