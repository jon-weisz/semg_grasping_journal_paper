\documentclass[10pt, journal]{IEEEtran}
\usepackage{amsthm,amssymb,natbib,url,amsmath}
\usepackage{caption}
\usepackage{subcaption}
\usepackage{multirow}
\usepackage{graphicx}



\newtheorem{theorem}{Theorem}



%%%Article information%%%%%%%%%%%%%%%%%%%%%%%%%%%%%%%%
\journal{Journal name}
%\volume{000}
%\issue{00}
%\copyrightline{$\copyright$ Copyright}
%\firstpage{1}
%\lastpage{13}
%\doi{doi number}
%\articletype{Article type}
%\pubyear{2015}
%%%%%%%%%%%%%%Article information%%%%%%%%%%%%%%%%%%%%%%%%



\markboth{Running head left side}{Running head right side}


\begin{document}

%\title{Assistive Robotic Grasping}

\title{Assistive Grasping with an Augmented Reality UI}

%\author{G.R. Thomson and C. Constanda\thanks{Corresponding author;
%e-mail: christian-constanda@utulsa.edu}}



\author{Jonathan Weisz, Alexander G. Barszap*, Sanjay S. Joshi*, Carmen Elvezio, Benjamin Shababo and Peter K. Allen\\
Department of Computer Science, Columbia University\\
*Department of Mechanical and Aerospace Engineering, University of California, Davis}

%\address{Department of Computer Science, Columbia University\\
%*Department of Mechanical and Aerospace Engineering, University of California, Davis}


\maketitle

\begin{abstract}
Assisting impaired individuals with robotic devices is an emerging and potentially transformative technology.  This paper describes the design of an assistive robotic grasping system that allows impaired individuals to interact with the system in a Human-in-the-Loop (HitL) manner, including the use of a novel cranio-facial electromyography input device. The system uses an augmented reality interface that allows a user to plan a grasp online that matches their task oriented intent. The system uses grasp quality measurements that generate more robust grasps by considering the local geometry of the object and the effect of uncertainty during grasp acquisition. This interface is validated by testing with real users, both healthy and impaired. This work forms the foundation for a flexible, fully featured HitL system that allows users to grasp known and unknown objects in cluttered spaces using novel, practical Human Robot Interaction paradigms that have the potential to bring HiTL assistive devices out of the research environment and into the lives of those that need them. 
% We present a series of experiments refining a Human-in-the-Loop assistive grasping system designed for low throughput, high noise interfaces such as surface electromyography of cranial and facial muscles. We use grasp quality measurements that generate more robust grasps by considering the local geometry of the object as well as how uncertainty will affect the proposed grasp. These new measures are integrated into an augmented reality interface that allows a user to plan a grasp online that matches their intent for using the object that is to be grasped.  This interface is validated by testing with real users, both healthy and impaired. This work forms the foundation for a flexible, fully featured HitL system that will allow users to grasp objects in cluttered spaces using novel, practical BCI devices that have the potential to bring HiTL assistive devices out of the research environment and into the lives of those that need them. 
\end{abstract}

%\keywords{Assistive Robotics, Grasp Planning, Electromyography,}

\section{Introduction}

\subsection{The Promise of Assistive Robotics}

With recent advances in robotics and computer vision, it is possible to imagine a
robotic  system  to  assist  people  with  severely  limiting  disabilities  in  activities  of
daily living, improving their quality of life. Common daily activities frequently require the user
to grasp an object stably in a context aware way. Complex hands and manipulators
increase the flexibility and grasping capabilities of a robotic assistant, but at the cost
of requiring more complex control of many simultaneous degrees of freedom (DOF). 

This work presents an assistive grasping system
for people  with  upper  limb  mobility
impairments using a human-in-the-loop paradigm that allows a disabled user to grasp objects
from a table using a novel, non-invasive surface electromyography (sEMG) based input device even in somewhat cluttered scenes. The novel device measures  only  a  single  differential  sEMG  signal  at  one
muscle site on the user. The system puts the user in control of a multi-phase grasping pipeline that includes object recognition, integrated pre-planned and on-line grasp planning with feedback  to  help  the  user  plan  robust  grasps  in  near  real-time.  
% Although progress in the robotics field has been swift, it is unlikely that truly independent operation of robots in situations where they will interact closely with objects, obstacles, and perhaps even other people in their environment will evolve in the immediate future. With the help of a human operator, it is possible to achieve robust and safe operation in complex environments. 
%Grasping is particularly important for many activities of daily living (ADL) that physically impaired %individuals need assistance performing, such as fetching food or communication devices.

The individuals with the greatest need for assistive technologies are those with severe impairments. Due to these impairments, they are often limited in their ability to provide input to an assistive device. Some current methods include sEMG, electroencephalography(EEG), eye-tracking, and sip-puff devices. In general, these devices are restricted to low bandwidth, noisy signals. Therefore, using these devices to control high DOF assistive grasping device poses many challenges.  Our solution is to combine intelligent online grasp planning with limited HitL assistance.

\subsection{The Challenge of Robotic Grasping}
Irrespective of the problems posed by limited input devices, robotic grasping is challenging for a number of reasons. Complex robotic hands have many DOFs, so the space of possible grasps is large and computationally expensive to explore. Standard approaches to planning in high dimensional state spaces are likely to fail with multi-fingered hands, especially as the grasp itself involves purposeful collision with the object, but most of the "near grasp" states will be overlapping the object in some way. Second, evaluating grasps involves several properties that are difficult to model, such as friction and closed chain kinematics. Many state-of-the-art analysis tools are only effective if the contact points can be perfectly predicted and the grasp acquisition can be perfectly controlled so that the object is not moved. Finally, robotic hands are extremely heterogeneous in terms of their physical size, the arrangement of their sensors, and their actuators, which makes designing generic grasp planning algorithms difficult.

In addition to all of these issues, in natural environments any set of grasps that is preplanned may overlap with obstacles in the environment or fail to grasp the object in a way that is well suited to the desired use of the object. Thus grasp planning algorithms must be fast enough to run online and be able to reflect the intent of the use of the grasp beyond simple stability.

This paper describes an sEMG driven assistive grasping platform integrating human in the loop planning in planning through an augmented reality interface. We present the iterative development process we have used to arrive at our final system, comparing different user interface paradigms and grasp planners presented in our previous papers addressing this problem (\cite{Weisz2012c,Weisz2013,Weisz2014}).  Then, we present new results from our final user validation study of our sEMG paradigm which addresses grasping in clutter with both known and unknown objects.
The key contributions of this work include:
\begin{itemize}

\item 
Design and comparison of three different user interfaces for assistive grasping.
\item
Integration  with  a  novel  sEMG  input  device  which  relies  on
only  a  single  muscle  site.
\item
A  new  UI  that  improves  the
disabled  user’s  ability  to  understand  the  scene  and  produce
correct grasps in complex, cluttered environments.
\item
Grasp reachability analysis and feedback to the user.
\item
Online assessment of the desired approach direction.
\item
Evaluation of this system on an impaired user in a remote location.
\item
A demonstration
that this new UI is descriptive enough for the user to operate
in an environment that they have never seen
\item 
The ability to grasp known and unknown objects amidst clutter
\item
Demonstration of the system on two different sets of hardware
\item 
Experimental results from both healthy and impaired users.
\end{itemize}


\section{Related Work}


\subsection{Human Computer Interfaces for Assistive Robotics} 
%\sububsection{Overview}
There is a long history of assistive robotic systems using electrophysiological signals as input, with work going back as far as  \cite{schmidl-65} and \cite{sherman-65}. In the time since, there have been myriad approaches and refinements of proposed of interfaces for disabled individuals with robotic assistive systems, and this work will not review even a small fraction of them. There are two ways of categorizing these systems. One way is to categorize a system by its input modality; i.e. whether it uses physical buttons or pointing devices, some external sensor of motion such as eye or hand trackers, or some specific electrophysiological signal such as EMG, electrooculagraphy (EOG), and EEG.  Within this category, modalities can be further divided by where the signals are recorded from. EMG can be recorded from distal muscle sites, which may be larger, easier to record from, and produce larger signals. However, more impaired individuals tend to maintain control over muscle functions closer to the head

Another way to categorize the systems is by the type of control they engender - whether the control is at a task level, allowing the user to designate what is to be done, or at a state level, allowing the user to specify joint angles, or end effector positions.  

This work presents a system at an intermediate control level, in which the user has some state level control that is task oriented. This requires an online planning system that generates robust grasps in real-time. Below we describe the different control paradigms used in related systems using human-robot interface (HRI) devices suitable for impaired individuals.  

\subsubsection{Direct Demonstration}
 The most intuitive, low level of control of a robotic arm involves having the robot arm directly mimic the motions of the user. It is possible to reconstruct a user's movements using distal limb surface EMG signals, as in \cite{Artemiadis2011,Castellini2009}.  This type of control allows the user to express their desires explicitly, allowing the user to specify how the arm is to avoid obstacles. This paradigm is not suitable for assistive robotic interfaces because many seriously impaired individuals have lost exactly the capability used as the control input to this type of interface.  
 
 \subsubsection{Joint Level Control}
If direct mimicry is impractical, the user can be given explicit control of joints of the robot. This generally imposes a much higher cognitive load on the user, as they have to attend closely to each joint. The movement of the joints are not directly related to user's goal of manipulating some object. Control of a manipulator through such an interface is generally not possible because they have many joints. The manipulator is generally controlled by simple open and close commands. For example, in \cite{Horki2011} hand opening/closing and elbow flexion/extension are controlled by EEG signals.

 For a prosthetic arm with essentially two degrees of freedom, this sort of control may be appropriate, but more degrees of freedom will strain the bit rate of these types of devices and their unreliability will make the coordination necessary to perform complex tasks in natural environments difficult or impossible. For example, to grasp an object using a six DOF manipulator with a gripper, the user must move in a straight line towards an object, or the gripper will not move straight and the finger may knock over the object while moving the palm into place.  This requires that the user simultaneously send coordinated signals to all six degrees in exactly the right ratios, or it may not move in anything close to a straight line.  

\subsubsection{End Effector Cartesian Control}
The main goal of a robotic manipulator is to interact with the world with some end effector. Giving the user direct control over the end effector location can be more intuitive, because the end effector location is the variable that the user most directly observes. This control scheme has been implemented using both invasive high throughput systems as in  \cite{Vogel2010},  and less invasive systems with lower throughput such as surface facial EMG and EOG, as in \cite{Postelnicu2011, Sagawa2005,Gomez-Gil2011,Ranky2010,Shenoy2008}.  Although this approach is similar to joint level control in requiring continuous attention to a relatively large number of degrees of freedom simultaneously, the user's control is directly in the task space. This allows the user to decouple the different controlled degrees of freedom.   

\subsubsection{Discrete State Level Control}
In discrete mode control, the user is able to switch between a set of predetermined configurations. This control paradigm is commonly used for control of robotic hands to swithc between a number of different hand configurations, as in \cite{Yang2009a,Woczowski2010,Ho2011,Cipriani2008,Matrone2011}.

 These control schemes represent a tradeoff between flexibility and simplicity of use. This tradeoff is especially important for as the complexity of the hand increases. Direct control over the fingers of a complex manipulator is not feasible because of the number of DOFs and the precision required to avoid collisions. Discrete level control allows more DOFs to be controlled safely with fewer inputs.

However, these schemes limit the user's flexibility to the preset configurations. Additionally, the user has to remember how to get to the configuration that they want to use at a given time, which may require multiple steps through a branching decision tree.  Because there is not necessarily an easy way of associating the path that they need to take in that decision tree with the goal they want to reach, these control schemes have a steep learning curve. 

\subsubsection{Task Level Control}
The key challenge of using noninvasive human robot interfaces is that the bit rate is low and that the input is somewhat unreliable. In addition, the user experiences limited feedback, which makes direct control difficult.  Under these conditions, it would seem intuitive that users would find task level control, where the user directs the robot on what to do but has little input as to how to, would be more effective. Indeed,  it has been shown that users find HRI control easier using even higher level, goal oriented paradigms \cite{Royer2011}, and we have begun to see work that attempts to exploit higher level abstractions to allow users to perform more complex tasks with robotic arms. 

In \cite{Bell2008}, EEG signals were used to select targets for pick and place operations for a small humanoid robot. \cite{Waytowich} used EEG signals to control pick and place operations of a 4-DOF St\"{a}ubli robot. \cite{M.BryanV.ThomasG.NicollL.Chang2011} presented preliminary work extending this approach to a grasping pipeline on the PR2 robot. In that work, a 3D perception pipeline is used to find and identify target objects for grasping and EEG signals are used to choose between them. In \cite{Muller-Putz2005}, grasping is decomposed to a four-phase pipeline where EEG signals are used to control transitions between phases. And in \cite{Scherer2011a}, the authors demonstrate an interface to navigate in two dimensions and select goals in a complex virtual environment and propose a hierarchical control scheme for learning high-level tasks dynamically. 

The drawback of this approach is that while the system presents the user with a set of high level choices, the user is not able to effect the process by which the choices are generated. In complex situations, the software agent may not present the user with appropriate choices. 

\subsubsection{Task Oriented Shared Control}
An emerging alternative to the purely task oriented approach is to blend end effector control and task oriented control. In this approach, the user's input demonstrates some approximation of the desired solution or constraint which an automated planner can make use of. \cite{Mulling2015} showed that this strategy can improve performance in grasping tasks even when using an invasive BCI device with relatively high bandwidth. In our work, we show that this strategy can allow non-invasive devices with much lower bandwidth to exercise similar performance in accomplishing complex tasks with high reliability. 

\subsection{The Eigengrasp Grasp Planner}
 In \cite{CiocarlieIJRR}, our lab introduced the Eigengrasp Planner, which allowed the user to grasp objects reliably by demonstrating only an approximate approach direction. In this work, we have expanded upon the Eigengrasp planner to show that task oriented shared control is a practical approach for allowing the flexibility of lower level control schemes with the ease of use of higher level task level control. 

\begin{figure}[hb]
\centering
	\includegraphics[width=.49\textwidth]{images_2/real_eg_grasp.png}
	\includegraphics[width=.49\textwidth]{images_2/simulated_eg_grasp2.png}
	\caption{An operator demonstrating the Eigengrasp Planner by manually guiding the robotic hand to guide the planner in the virtual environment(\cite{CiocarlieIJRR}).}
	\label{fig:egplanner_demo} 
\end{figure}

The Eigengrasp Planner allows a user to interact with an online grasp planner in a virtual environment to plan grasps online in real-time. The user is given control of a virtual representation of the hand which they use to indicate approximately where they would like to grasp the object. A grasp planner runs in the background and presents the user with a set of options for completing the grasp. 
This strategy requires a responsive planner that can handle the complex problem of grasp planning in near real-time. To make this computationally tractable, Eigengrasps were introduced, a dimensionality reduction technique in which control of the hand is mapped to principle components identified in human grasping studies. With this dimensionality reduction, stochastic sampling techniques can be used to generate reasonably good grasps in real-time using relatively simple grasp quality metrics.
 
The quality metric that is used by the planner evaluates a projection of the desired contact points on the hand to the target object. This projection provides a smooth energy gradient in regions where the hand is not in contact with the object. When good candidates are found, the planner simulates completing the grasp by approaching the object along a pre-specified direction orthogonal to the "palm" of the robot hand and then closing the fingers. 

This approach has a number of practical advantages. The nature of the optimization approach, which gradually moves towards lower values of the quality function, produces solutions where nearby finger contacts will also provide similar quality grasps. Grasps where the qualities of nearby configurations are much poorer will have narrow basins of attraction that are less likely to be found. This implies a certain amount of robustness to small displacements and occlusion of the object from nearby clutter during grasp acquisition. The planner is easy to generalize because the only robot specific parameters are the state space reduction strategy and a set of desirable contact locations, which can be easily specified for any given robot. 

In \cite{CiocarlieIJRR}, the planner was demonstrated by having the operator manually move the end effector in real-time (see Figure \ref{fig:egplanner_demo}). This is analogous to an extremely high bandwidth, low noise interface with perfect knowledge of the environment. In this work, we have fleshed out this demonstration to more realistic, complex situations. This required development of a full robotic grasping platform that can handle cluttered scenes, realistic input devices appropriate for disabled people, and an augmented reality user interface. The development of such a system is the central challenge addressed in this paper.  


\subsection{ Roadmap of this paper}

To address this challenge, we have iterated through four designs of our assistive robotics system, denoted System 1-3.  Section \ref{sec:pipeline_1} describes user experiments using \emph{System 1} with five unimpaired subjects. In Section IV, we describe \emph{System 2} which integrates the novel EMG interface device and allows grasping in cluttered scenes, and test its efficacy with an impaired user in a remote location. In Section V, we describe \emph{System 3} which uses a different set of hardware and also improves the speed and reliability of the user interface and demonstrates it on a cohort of unimpaired subjects. 




















\section*{System 1: An HRI Grasping Platform Prototype}
\setcounter{subsection}{0}
\label{sec:pipeline_1}
\begin{figure}
	\centering
	\includegraphics[width=.8\columnwidth]{ui_1.png}
	\includegraphics[width=.8\columnwidth]{images_4/overview_pipeline.png}\\
	\caption{ \emph{Top:} An annotated screenshot of the prototype grasp planning user interface in GraspIt!. During online planning, the user is presented with an augmented reality view of the target object and three renderings of the hand interacting with the scene. The \emph{Planner Hand}, which is the most transparent hand, demonstrates the current state of the planner. The \emph{Input Hand} which is of intermediate transparency, is the hand through which the user directs the planning system. Here you can see the rotational guides which allow the user to visualize their available control directions. The \emph{Solution Hand}, which is fully opaque, demonstrates the best grasp currently available. This is the grasp which is closest to the approach direction that the \emph{Input Hand} is demonstrating and which also has the best grasp quality. \emph{Bottom:} The four phases of a basic grasp planning task. Breaking the task into phases allows customization of the user interface for each phase independently to make optimal use of low input bandwidth.}
	\label{fig:overview_pipeline}
\end{figure}
\renewcommand*{\theHsection}{chX.\the\value{section}}
In \cite{Weisz2012c}, we developed a prototype BCI enabled grasping platform through which we outlined a general strategy for an online assistive grasping system. 
The grasping task can be decomposed into a four subtasks: Target object identification and localization, generation of grasp plans, picking an optimal plan, and executing the plan on the robot. Each subtask can be fulfilled by different modules which benefit from different user interaction strategies. By decomposing the tasks into explicit phases of a pipelined process, as in Figure \ref{fig:overview_pipeline}, we can optimize user's interaction for each phase to make the best use of input modalities with limited bandwidth while guiding the grasping platform. 
Although fully automated approaches for each of these subtasks have been the subject of extensive and ongoing research, integrating user input to create a shared-control environment that uses as much input as the user is able to supply is still a relatively unexplored field.

 There are many possible paradigms for integrating HRIs with a shared-control assistive robotic device. Traditional EMG and EEG setups are expensive and difficult to deploy. In this work, we wanted to explore the boundaries of what can be achieved with devices that are more practical for a real world assistive device, both in terms of convenience and cost. We experimented with two low cost devices for detecting EMG, the Emotiv Epoc (Emotiv Systems Inc., San Francisco, CA, USA) and a custom device described in Section \ref{sec:semg_hardware}. 

Putting the human in the loop when planning and executing the grasp in real-time fundamentally changes the nature of the problem as compared to a fully automated system. The key challenge becomes conveying information to the user effectively about the state of the system and then using the low bandwidth information gained from the user efficiently. This requires careful design of the interfaces provided to the user and of the control scheme for inferring intent from the user's input. Additionally, in order to present the user with reasonable grasping options, we need to extend the existing grasp stability analysis to deal with the most common problem that arises in unstructured environments, object localization errors due to sensor noise (See \cite{Weisz2012} for an analysis of the effect of sensor noise on the Eigengrasp Planner). 

\subsection{Prototype Design Components}
\subsubsection{Hardware Components}The manipulator arm for the initial prototype was composed of an industrial St\"{a}ubli TX60L robotic arm and a BarrettHand gripper. The object localization system was based on point clouds captured by a Microsoft Kinect depth camera. We sought an input device that might be representative of what could be achieved by a low cost BCI device in order to evaluate a user interface under realistic conditions of error and throughput. The Emotiv Epoc was chosen because of the convenience of its wireless form factor and relatively low cost. 

\subsubsection{EMG Input Processing}
The Emotiv Epoc comes with three built-in signal processing modalities designed to detect emotional affect, facial movement, and EEG evoked responses. Combining these classifiers, we were able to derive a training paradigm for detection of four facial gestures robustly. For details see \cite{Weisz2013}.

\subsection{User Interface}
We augmented the Eigengrasp Planner GUI in the GraspIt! simulator (\cite{Miller2004}) with a visualization of the grasp planning scene that includes a number of guides and fiducials that allow the user to guide the planner fully inside the simulator. The augmented grasp planning scene is illustrated in Figure \ref{fig:overview_pipeline}. 

The four facial gestures captured from the Epoc are treated as discrete on/off signals. We found some facial gestures, such as eyebrow raising, to be easier to maintain than others such as winking. These were assigned to control signals whose duration controlled some continuous value, such as position of the end effector along the guides. Two of the gestures are mapped to "Yes" and "No" inputs at decision points, while the remaining two control the rotation of the \emph{Input Hand} along the guides shown in Figure \ref{fig:overview_pipeline}. A video demonstrating the operation of this prototype can be found here: \url{https://youtu.be/_b5ecKbIHBQ}.

\subsection{Software Platform}

\subsubsection{Planning and Kinematics}
Planning for the motion of the arm is done in OpenRAVE using a bidirectional random tree planner in \cite{berenson-cbirrt}, and small linear motions near the object are planned using the TX60L's built-in inverse kinematics planner.

\subsubsection{Recognition System}
\label{sec:rec_system}
 We use the \emph{Model Ransac} method described in \cite{EfficientModelRansac} to identify and localize the target object in the scene. This method generates features from pairs of oriented points on the surface of the object. Prospective models are processed off-line and put into a hash table. Features are sampled from the sensor data and tested for collision in the hash table. If a sufficient number of collisions occurs with points on the same model, a variant of RANSAC is used to test the hypothesis that a set of points in the sensor data corresponds to a particular model at a particular location. This method has demonstrated good robustness and is extensible to multi-object scenes. 

\subsection{Evaluation and Shortcomings}
Using this grasping platform, the experimenters were able to develop an interface for the Eigengrasp Planner and access it through a noisy, EMG based interface. However, we found that users with less experience and patience were not able to interpret the information in the grasp planning scene. In order to run user experiments, we needed a more informative UI and some faster planning options.

\section*{System 2: Adding Visual Feedback}
\setcounter{subsection}{0}
\renewcommand*{\theHsection}{chX.\the\value{section}}
After initial experiences with \emph{System 1}, we noted several shortcomings in the design of the user interface. With additional visual cues, we were able to add several features and allow non-expert users to successfully navigate the grasping pipeline. We added elements to the UI that are illustrative of the current state of the planner and the available options at each phase of the planner (see Figure \ref{fig:ui_2}). Additionally, this extended UI gives feedback about the position of localized object with respect to the sensed environment, allowing the user to interact directly with the localization system. With these modifications, we also enable the user to grasp "novel" objects that are not in the object database and use our pre-planned grasp database. 

\subsection{Handling Novel Objects}
\begin{figure}
	\centering
	\includegraphics[width=.6\columnwidth]{ui_2_flipped.png}
	\caption{\emph{System 2 Interface - Online Planner Phase}. The user interface contains three windows: The main window containing three labeled robot hands and the target object with the aligned point cloud, the pipeline guide window containing hints for the user to guide their interaction with each phase of the planner, and the grasp view window containing rendering of the ten best grasps found by the planner thus far.}
	\label{fig:ui_2}
\end{figure}
In order to handle objects that are not in the recognition system, we rely on the stochastic nature of the planning and recognition system and the discernment of the user. When automated systems fail, the user can reject the proposed solutions and wait for another. The parameters of the object recognition system can be tuned to recognize objects with similar parts by increasing the allowed error in the hypothesis testing stage of RANSAC. An example of this alignment can be seen in Figure \ref{fig:alignment_1}. In order to allow the user to discern how well the detected object aligns to the true geometry of the novel object in the scene, the UI was modified to include a down-sampled point cloud from the depth camera. The user is responsible for rerunning the vision system until they see a reasonable alignment of the sensor data and detected model. This interaction also comes into play in the grasp planning phase, in which we rely on the user to reject grasps that may seem appropriate for the detected model but do not fit the actual unknown model well. 


\begin{figure}
\begin{subfigure}[b]{.45\columnwidth}
\centering
\includegraphics[width=.95\columnwidth]{alignment_1.png}
\caption{Point clouds with RGB texture from the vision system. On the left is a flashlight along with its aligned point cloud in white. On the right is the point cloud of a juice bottle along with the best model from the vision system’s object database, a shampoo bottle, in white.}
\label{fig:alignment_1}
\end{subfigure}
\begin{subfigure}[b]{.45\columnwidth}
\centering
\includegraphics[height=0.95\columnwidth]{unknown_objects_1.png}
\caption{The two objects which are aligned on the right of the figure above. The shampoo bottle is in the object database. The juice bottle is not. The two are roughly the same width, and this shampoo bottle can be an appropriate proxy for the juice bottle in the planner.}
\label{fig:unknown_objects_1}
\end{subfigure}
\caption{Results of the object recognition system with known and unknown objects.}
\end{figure}


\subsection{Incorporating a Grasp Database}
One useful aspect of mapping the object in the scene to a set of objects from a database is that we can also pre-plan a set of grasps for each object. This provides a method for the user to skip the slower online planning phase, since we will already rely more on the user and less on the automated grasp quality analysis. 

Using a grasp database also allows us to manually design good grasps for particular affordances that are difficult for an automated planner to recognize. Figure \ref{fig:manual_grasp_1} demonstrates such a grasp, which is realizable by the BarrettHand only because the soft plastic surface of the object deforms during grasp acquisition to allow the finger to pass through the hole in the handle region of the bottle.  In experiments, this grasp was successful 100\% of the time.  Capturing this behavior in a simulator would require modeling dynamic object deformations. Currently, accurate simulations of such properties are too slow for sampling based planners, and so human annotation of such grasps is necessary. 

To generate the grasp database, we ran the Eigengrasp Planner off-line six times for twenty minutes each with the approach direction of the palm aligned to the major axes in the positive and negative directions, using the best grasp from each direction in the database. If there were fewer than ten grasps in the database, including manually inserted grasps, then the highest quality grasps were selected from among all of the available grasps until a full ten are available. 

\subsection{Grasping Pipeline}
\begin{figure}[th!]
	\centering
	\includegraphics[width=1\columnwidth]{pipeline_2.png}
	\caption{The phases of the modified grasping pipeline incorporating more control over the vision system and the grasp database. If the user chooses the \emph{nth} grasp from the database, \emph{n+4} user inputs are required. If none of the grasps are suitable, the online planner can be invoked with a few simple inputs to refine one of the grasps further.}
	\label{fig:pipeline_2}
\end{figure}
\label{sec:pipline_2}
These improvements result in a more flexible, but complex grasping pipeline. In this pipeline, there are three additional phases before the  \emph{planning-review-execute} pipeline outlined previously in which the augmented visualizations are used to provide the user with extra flexibility in initializing the planner. These phases allow the user to first control the vision systems object detection and localization, then review the set of available grasps retrieved from the pre-planned database, and finally to chose whether to activate the online refinement or simply execute the retrieved grasp. The phases of the pipeline and transitions between them are outlined as a state machine in Figure \ref{fig:pipeline_2}. Notice that these modifications create two possible paths to the final phase of grasp execution, trading off complexity for potential speed improvements. 

To help the user manage this complexity, we added a "Guide Window" that displays the transition table for the current phase, as shown in Figure \ref{fig:ui_2}. Throughout the pipeline, two of the facial gestures are associated with transitions between phases of the pipeline, while the other two provide direction to the online planner. In the "Guide Window", the left side of the window shows the current phase in yellow, the center of the window shows the result of the Gesture 1 transition in red, and the right side of the window shows the result of the Gesture 2 transition in green. 

Below the "Guide Window," we visualize the top ten grasps available to the planner. This allows the user to judge their options as they are retrieved from the database or repopulated by the online grasp planner. 


\begin{figure}
\centering
\includegraphics[height=.2\textheight]{manual_grasp_1.png}
\caption{This handle grasp for the detergent bottle is not a force closure grasp, but when chosen by the subjects in our experiments it succeeded 100\% of the time. Adding a grasp database allows such semantically relevant grasps to be used in our system.}
\label{fig:manual_grasp_1}
\end{figure}


\subsection{Experiments}
In order to test the efficacy of our system, we recruited five healthy subjects to participate in an experiment to use the system to lift three objects from a table.  All testing was approved by the Institutional Review Board of Columbia University under Protocol AAAJ6951. The results of these experiments were published in \cite{Weisz2013}, and a video illustrating the experiments can be found at \url{https://youtu.be/3YnbxVsJKs0}.

\subsubsection{Task}
Each subject was asked to grasp and lift three objects using an Emotiv Epoc as input. Two of the objects, a flashlight and a detergent bottle, were in the database and available to the vision system. One of the objects, a small juice bottle, was novel. Each subject was asked to perform two grasps, one from the top of the object and one from the side of the object. Each grasp was repeated three times. For the novel object, subjects were simply asked to grasp the object five times, irrespective of direction.


\subsubsection{Training}
\label{sec:emotiv_training}
The subject was asked to perform each facial gesture ten times each to train the Emotiv Epoc classifiers and choose reasonable parameters for the classifiers. For details, see \cite{Weisz2013} To train the subject to perform the task, the subject was asked to perform the task twice in the virtual environment without executing the final grasp on the arm. 

\subsubsection{Results}
The results of the experiments are reported in Table \ref{tab:results_2}. For each subject, we report the mean time to completion and fraction of successful attempts for each grasp. Time to completion is measured from the end of the object identification phase to the beginning of the execution phase, which is the time taken to plan the grasp. Overall, the average planning time was 104 seconds on the known objects and 86 seconds on the unknown object. The mean success rate was 80\%, demonstrating that this system is effective in allowing the user to plan and execute a reasonable grasp for these objects.
In these experiments, grasps from the side demonstrated significantly more robustness and lower planning times than grasps from above. The grasp database contained only one grasp from above for each of these objects, and this grasp was a fingertip grasp which may be sensitive to pose estimation error, which resulted in longer planning times while the subjects searched for a better grasp. In general, grasping roughly cylindrical objects such as the top of the detergent bottle from above is somewhat problematic for the BarrettHand due to its configuration and the low friction of its fingertips. In contrast, subjects were able to find a reasonable grasp from the side of the object among the grasps pulled directly from the database. The difference in planning times reflects the benefit of integrating the off-line planning phase.

\begin{table}
\centering
\begin{tabular}{ | c | c | c | c | }
\hline
Grasp & Subject & Successes & Mean Time(s) \\ \hline 
\multirow{5}{*}{Flashlight Side} & 1 & 3/3 & 125 \\ 
& 2 & 3/3 & 53 \\ 
& 3 & 2/3 & 103 \\
& 4 & 3/3 & 95 \\
& 5 & 3/3 & 82 \\ \hline
\multirow{5}{*}{Flashlight Top} & 1 & 3/3 & 132 \\ 
& 2 & 2/3 & 75 \\ 
& 3 & 2/3 & 96 \\
& 4 & 3/3 & 93 \\
& 5 & 2/3 & 125 \\ \hline
\multirow{5}{*}{Detergent Bottle Side} & 1 & 3/3 & 75 \\ 
& 2 & 3/3 & 57 \\ 
& 3 & 3/3 & 106 \\
& 4 & 2/3 & 82 \\
& 5 & 3/3 & 75 \\ \hline
\multirow{5}{*}{Detergent Bottle Top} & 1 & 1/3 & 151 \\ 
& 2 & 2/3 & 114 \\ 
& 3 & 2/3 & 142\\
& 4 & 2/3 & 161 \\
& 5 & 3/3 & 145 \\ \hline
\multirow{5}{*}{Novel Bottle} & 1 & 3/5 & 132 \\
&2 & 4/5 & 63\\
&3 & 4/5 & 95\\
&4 & 4/5 & 91\\
&5 & 4/5 & 50\\
\hline\end{tabular}
\caption{Results from Experiment 2}
\label{tab:results_2}
\end{table}

\subsection{Discussion}
During the experiment, we found that it was necessary to re-wet the electrodes of the Epoc many times during the experiment. Additionally, for three of our subjects it took more than an hour to find the right thresholds and position for the headset. After the experiment, subjects were asked to describe their discomfort during the experiment and their level of control. Subjects reported little discomfort initially, but were frustrated with the difficulty of getting the Epoc to recognize their intended actions, especially with false negatives making it difficult to continue to the next phase of the pipeline at will. This led to overemphasis of the facial gestures, which caused muscle fatigue. In spite of this frustration, subjects were able to complete the task. 

However, these difficulties pose a major problem in testing this system on a disabled subject. Since they are dependent on caretakers, an indeterminately long setup time poses a major problem in performing studies with that population. Another major source of issues is the lack of online reachability testing in the grasp planner. In these experiments, we placed the object carefully to avoid having the planner find unreachable grasps. In more cluttered scenes, this issue would be problematic. Finally, users reported that the pipeline guide window was difficult to read while focused on the task.

In the next section, we describe a different interface device that is designed specifically to measure facial EMG signals, along with some of the changes we made to the user interface to address the concerns subjects expressed during this experiment.


\section*{System 3: Novel sEMG Device with Impaired User Study}
\setcounter{subsection}{0}
\renewcommand*{\theHsection}{chX.\the\value{section}}
\subsection{Surface EMG Recording}
\label{sec:semg_hardware}
We found a number of practical issues with the Emotiv Epoc as an input device. The two most important were the impracticality of the form bulky factor of the device and the difficulty of configuring and maintaining its electrodes to work correctly. To address some of these issues, we adopted a novel input device under development at UC Davis which is designed to be used by severely impaired individuals. This device has an extremely noninvasive profile, requiring only a single sEMG recording site behind the ear. 

The muscles behind the ear are are innervated by nerves that come directly from the brain stem, without ever entering the spine. Even individuals with the most severe spinal cord paralysis can still access these muscles. Although some individuals are able to independently move their ears, we have found that even individuals that cannot move their ears can learn to activate the muscles in that region without achieving overt motion when they are given visual feedback. This activation produces signals that can be detected by electrodes mounted on the surface of the skin. 

A series of works including \cite{joshisensor,JoshiTwoDimCursor,JoshiEPLPilot} have shown that the input device can record two simultaneous channels from a single recording site. This is achieved by training the subject to modulate the activation of the muscles near the recording site so that they can voluntarily control the power in two separate frequency bands. These two independent degrees of control are used to drive a cursor which selects options by hitting targets on a screen. In those works, the authors produced different user interfaces such as a UI for allowing a disabled individual to control a television. 

The general methodology is outlined in Figure \ref{fig:semg_processing_pipeline}. The single sEMG signal is first processed through a 60 Hz noise filter to remove noise from the AC power supply. It is then run through two different band pass Buttersworth filters to extract two separate signals. The bands are then linearly combined to compute the \emph{x} and \emph{y} cursor positions. This linear combination is necessary to generate independent control channels since there are no perfect band pass filters, and the subject may not be able to completely the frequency bands. 

The total powers of two different frequency bands of the single sEMG signal were computed using two band pass filters for 80-100 Hz (Band 1) and 130-150 Hz (Band 2). These bands were selected ad-hoc, based on previous experience. The output of the two filters produced comparable powers during maximum voluntary contraction. The filter outputs were combined linearly as described in Equation \ref{eq:equation_1}. 


\begin{equation}
\centering
\left[ \begin{array}{c} x\\ y \end{array} \right] = \left[\begin{array}{c c} 1.75 gain_1 & -.75 gain_2 \\ -.75gain_{1} &  1.75gain_{2}\end{array} \right]\left[\begin{array}{c}Channel_1 \\ Channel_2\end{array}\right]
	\label{eq:equation_1}
\end{equation}


Without this transformation the cursor could not reach points along the \emph{x} or \emph{y} axis as there can never be zero power in either of the frequency bands. The gains for each band are set for each subject after a short calibration procedure, as described in \cite{JoshiTwoDimCursor} to establish the subject’s comfort level maintaining a large enough voluntary muscle contraction to move the cursor to any part of the screen.

The sEMG signals are collected from the PA muscle with two surface Ag-AgCl cup electrodes connected to a model Y03 preamplifier (www.motion-labs.com) with input impedance higher than $10^{8} \Omega$, 15-2000 Hz signal bandwidth and a gain of 300. The electrodes were placed behind the subject’s left ear along the axis of the muscle with approximately 1.5 cm inter-electrode distance (see Figure \ref{fig:user-semg}. A third electrode was placed on the elbow as a reference. The cup electrodes were the type EL254S from Biopac Systems Inc. held in place with Ten20 conductive paste. 

To adapt this system to our use, we added some additional smoothing steps similar to those in \cite{vernon2011}. The cursor position is further filtered through a low-pass filter with a cutoff frequency of .5 Hz. This produces a new position at 4 Hz. To smooth the visualization of the cursor motion, we linearly interpolate 7 intermediate positions between each successive update, increasing the refresh rate of the visualization from 4 Hz to 32 Hz. This makes the system feel significantly more interactive, at the cost of a .25 second delay between the calculated position and the visualization.

\begin{figure}
\centering
\includegraphics[width=.99\columnwidth]{semg_processing_pipeline.png}
\caption{The single sEMG signal is first processed through a 60 Hz noise filter. It is then run through two different band pass Buttersworth filters to extract two separate signals. The bands are then linearly combined to compute the x and y cursor positions.}
\label{fig:semg_processing_pipeline}
\end{figure}

\subsection{sEMG GUI}
To send signals to the grasping system, the user controls a cursor to hit one of four targets, as illustrated in Figure \ref{fig:semg_ui_blank}. Each target represents a different input options. During grasp planning, they are overlaid on the augmented reality display. The user begins in a rest area and moves the cursor to one of the targets. When the target is hit, the cursor changes colors to reflect the user's selection. The user returns the cursor to the rest area, at which point the input option selected is activated. After a selection, the other targets are disabled for four seconds. If an unintended target is selected, the user can force the selection to \emph{timeout} by avoiding rest for these four seconds, canceling the selection. 

We map these inputs to following a similar strategy to that used for each facial gesture in \emph{System 2}. For the red and green inputs, denoted input 1 and input 2, the input is activated a single time when the user returns to rest. For inputs 3 and 4, the magenta and black targets respectively, the activation is sent continuously until the user exits the rest area again. This allows the user to exert near continuous control over the approach direction. 

\begin{figure*}
\centering
\includegraphics[width=.99\textwidth]{semg_interface_blank.png}
\caption{The sEMG Interface: (a) The user interface is composed of 4 targets overlaid on the grasping scene. Target 1 usually signals acceptance of the current option. Target 2 toggles the next option. Targets 3 and 4 provide input to the planner.(b) Hitting one particular target changes the color of the cursor to reflect the selection and makes the other targets unavailable. (c) If the user does not return to the rest area after a few seconds, the selection times out and is deselected and all targets become available again for selection. }
\label{fig:semg_ui_blank}
\end{figure*}


\subsection{Handling Cluttered Scenes}
In addition to an improved input device, we extended our grasping system to handle more realistic scenes, including some amount of clutter. In this work, we will define ``clutter'' as objects being in close enough proximity that many of the grasps for the objects may collide with other nearby objects, but that they are not actually in contact with one another. We did not handle the problem of singulation, which is a specialized manipulation designed to separate objects which are too close together for the fingers to surround the object without colliding with other objects. As such, we tested grasp planning scenes where there was at least 5cm of empty space between each object. 

Handling cluttered scenes brings up a number of challenges. First, it slows down the online planning phase. There are many fewer possible grasps and the obstacles divide the state space into discontinuous regions, which creates more local minima in the value of the quality function, which slows down convergence. Additionally, adding more geometry to the planning scene slows down collision detection, which is a bottleneck for grasp planning. Second, many of the grasps produced by the planner may not have a reachable path to grasp. In our previous systems, we made the optimistic assumption that most grasps were reachable, but in clutter this is no longer a valid assumption. Third, with more objects in the scene, there is more visual clutter and it is more difficult to produce a useful visualization.

In order to address the first two issues, we implemented an online reachability test which the user can clearly interpret. When good grasp candidates are found, \emph{System 3} checks that an entire valid trajectory can be generated using the CBiRRT planner described in \cite{berenson-09}. Unreachable grasps are placed at the end of the list of grasps and colored red in the grasp preview window (see Figure \ref{fig:semg_ui_b}). This allows the user to see that progress is being made even when no new reachable grasps are being generated.

We maintain the list of unreachable grasps so that we can reject nearby grasps without running more computationally expensive analyses. The valid grasps are ranked by their distance to the demonstration hand and alignment to its approach direction. This makes the planner more responsive in cluttered scenes. The list is re-sorted as the demonstration hand is moved.

The results of the reachability test are also used to train a nearest neighbors classifier. When the user moves the demonstration hand, we find the five grasps for which the normal of the palm of the hand is closest to the normal of the demonstrated pose. If at least 50\% of these grasps are unreachable, we designate the current demonstration pose as being in an unreachable region, which is indicated to the user by highlighting the demonstration hand in the planner interface in red. These measures are crucial for a naive user that is not familiar with the kinematics of the robot arm and may not have the intuition that the region they are trying to grasp from is not within the robot's workspace.

\subsection{GUI Modifications For Clutter and sEMG Interface}
 A number of changes were made to the user interface to accommodate both the added visual complexity of overlaying the sEMG control interface on the planning scene and the added difficulty of interpreting the multi-object scene. We redesigned the UI with a cleaner look and feel that implements a number of new features. 

The \emph{System 3} interface layout is outlined in Figure \ref{fig:semg_ui_a}, which illustrates the UI presented during the object selection phase. First, the point cloud displayed in the scene has been upgraded to a higher resolution, color point cloud. This change allows the user to discern the target object more effectively in the cluttered scene. It also allows the user to exercise more judgment in interpreting the scene, since they may not be physically present to observe it first hand. Second, we display only three grasp options instead of 10 to reduce the visual clutter. This also allows us to enlarge the presentation of the grasp so that the user can more easily discern how the hand may interact with the rest of the objects in the scene. Third, we moved the grasp preview window to the side of the screen and modified the way that the UI generates the view to share the aspect ratio and alignment of the depth camera so that all detected objects are visible and that the user's intuition is unimpeded by deformations due to the aspect ratio. Fourth, we removed all of the window decorations and grasp metric displays, as the subject is not expected to be able to interpret them correctly. Overall, this provides a much cleaner, streamlined view more suitable for non-expert users. 

Unfortunately, additional visual clutter is introduced by the addition of the sEMG user interface. This interface is rendered as a translucent layer on top of the grasp planning scene, allowing the subject to see both at the same time. The scene is chosen so that the relevant objects are centered in the scene. While the user is not actively using the cursor it remains in the lower left of the screen and the main grasp planning scene remains unoccluded. 

We placed targets 1 and 2 in opposite corners of the screen because these inputs control the progress of the user through the grasping pipeline, and we wish to minimize confusion between an accidental selection of these two options. The middle two options modify the demonstrated approach direction, and accidental selections have minimal impact. 


\begin{figure}
\centering
\begin{subfigure}[t]{\columnwidth}
\centering
\includegraphics[width=.99\columnwidth]{ui_3_a.png}
\caption{\emph{System 3 Interface - Object Selection Phase.} The subject is able to see the planning scene in the main UI window. The window on the bottom tells the user the current phase and what the green and red inputs will do in this phase. In this phase, the subject sees the point cloud and hits the red target until the object they wish to grasp is highlighted in green. Then they hit the green target to proceed to the next phase.}
\label{fig:semg_ui_a}
\end{subfigure}
\begin{subfigure}[t]{\columnwidth}
\centering
\includegraphics[width=.99\columnwidth]{ui_3_b.png}
\caption{\emph{System 3 Interface - Initial Review Phase.} After the subject selects the object, the Grasp View pane on the right is populated with a set of grasps from a database. Grasps that are reachable appear on a green background, while unreachable grasps are red. The topmost grasp in the grasp view window is the currently selected grasp, which is rendered in the planning scene with the planner hand.}
\label{fig:semg_ui_b}
\end{subfigure}
\end{figure}

\subsection{System 3 Pipeline}
\label{section:pipeline-v3}
\emph{Initialization:} The subject is shown presented with a view of the scene from the perspective of the depth camera. The user sends input 1 to activate the object recognition system. If the recognized objects align well with the point cloud sent, they can accept the results with input 1. If not, they can rerun the recognition system with input 2.

\emph{Object Selection:} The first detected object is highlighted in green as the target object. To select an object as a target, the user sends input 2. To cycle to the next object in the recognized object list the user sends input 1. The non-target objects are all highlighted in red. The non-target objects are replaced with lower resolution models when a target is selected, which makes the planning phases faster. 

\emph{Initial Review:} The user is presented with a list of pre-planned grasps from a precomputed database. This phase has been modified from \emph{System 2} to present the user with a clearer visualization and reachability information. As they iterate through the grasp list, the grasp in the middle and bottom rows shift up and the next grasp in the list moves in to the bottom position. Moving the demonstration hand will also cause the grasp list to re-sort bringing the new approach direction to the top of the list. Reachable grasps are presented on a green background, while unreachable grasps are presented on a red background. The user sends input 1 to increment through the grasp list. When the user finds a reasonable looking grasp, they send input 2 to select the grasp.

\emph{Planner Initialization:} The user is presented with the choice to either accept the grasp from the previous phase with input 1, proceeding straight to the Grasp Choice Confirmation phase or they can send input 2 to refine their chosen grasp further.

\emph{Grasp Refinement:} The online planner runs, presenting the user with updated options as new grasps are found. The new grasps are displayed with a white background in the grasp preview pane on the right side of the screen while they are being analyzed for reachability. Moving the demonstration hand causes the planner to generate grasps from the demonstrated approach direction. Sending input 2 stops the planner and proceeds to the Final Grasp Review phase.

\emph{Final Grasp Review:} The user is presented with three grasps at a time, which they can iterate through to select one that represents their intent. The user sends input 1 to switch to the next grasp on the grasp list. They send input 2 to select the current grasp. 

\emph{Grasp Choice Confirmation:} The user sends input 1 to go back to the Grasp Refinement phase and input 2 to send the grasp for execution on the robot.


\subsection{Validation}
To validate this system, we recruited a male 30 year old impaired subject with limited upper limb mobility due to a C3-C4 spinal injury. All testing of \emph{System 3} was approved by the Institutional Review Board of the University of California, Davis under Protocol 251192-10. This subject had previous experience with the sEMG device, but had not been trained on this interface. For this work, we measured the activity in the subject's PA muscle to avoid the need to shave the subject's hair. The subject was recruited and trained at the UC Davis site, and operated the robot without ever having any interaction with it or the experiment site in the real world at the Columbia University Robotics Group lab. The setup is shown in Figure \ref{fig:user-semg}. The top of the figure shows the subject with the device attached and using the grasp planning system. The bottom left image is a closeup of the device, which demonstrates how low profile and minimalistic this device is. The bottom right shows the target scene in the robot workspace, with three container objects. The results of this experiment were published in \cite{Weisz2014}, and a video illustrating the experiment can be found at \url{https://youtu.be/tRPXmb9yUbA}.
\begin{figure}
\centering
\includegraphics[width=.8\columnwidth]{user_semg.png}
\caption{An impaired subject in the UC Davis RASCAL
lab (top) operating our sEMG-Assistive Grasping interface
to grasp a shaving gel bottle in the Columbia Robotics
Group Laboratory (bottom right). The two small black clips
behind the subject’s ear (bottom left) are surface EMG
electrodes (used in differential mode) to detect activation of
the Posterior Auricular (PA) muscle to direct the system to
pick up the object in this multi-object scene.}
\label{fig:user-semg}
\end{figure}


\subsubsection{Task}
Due to limitations on the impaired subject's time, we were only able to complete three trials using the system. In these three trials,  the subject was asked to pick up an object from a cluttered, multi-object scene. In the first two attempts, he was asked to use the online planner to refine one of the pre-planned grasps. In the first attempt, he grasped the laundry detergent bottle. In the second attempt, he grasped the shaving gel bottle. In the third attempt, he was asked to grasp the detergent bottle using one of the pre-planned grasps directly from the grasp database. Other than the image in the planner interface, the subject was not given any information about the objects he was to grasp. However, they are all well known, household objects, so the subject can be expected to have some implicit idea of the weight and friction properties of the object. 


During the task, the subject reported which target he was trying to reach and we tracked the number of mistaken target activations, which would lead the user to loop back through that part of the pipeline. After the grasp is selected, the target object is lifted off of the table automatically so that the user can see whether the grasp is stable. If no part of the target object remains on the table, we consider the trial a success.

\subsubsection{Training}
To familiarize the subject with the interface, we demonstrated the pipeline two times with the subject just watching and asking questions along the way. We then went through the pipeline with the subject two more times while verbally instructing him on which target to hit while the experimenter controlled the cursor with a computer mouse. This allowed the subject to familiarize himself with the pipeline and navigate their way through it without having to also focus on the task of hitting targets with the sEMG interface. Once he appeared to be conversant with the system, we turned over control to the subject's sEMG interface.

\subsubsection{Results}
The results of the experiment are shown in Table \ref{tab:semg_table_1}. The subject was able to grasp the objects successfully on all three attempts. On average, it took the subject 694 seconds to grasp each object, including about 60 seconds for the vision system to detect the objects in the scene. There were an average of 25 timeouts, and 1 mistakenly selected targets per attempt. Timeouts are an expected part of this interface, which allows the user to re-select their intended target if the initially selected target is incorrect. Occasional mistaken selections are also expected, and the pipeline is designed to be robust to these errors, allowing the user to go back to the previous step where necessary to correct mistakes. Several mistakes in a row are necessary to actually realize mistaken actions on the robot.


\begin{table*}[ht!]
\centering
\begin{tabular}{|c|c|c|c|c|}
\hline
Grasp & Time(s) & \#Inputs & \#Timeouts & \begin{minipage}{.6in}\centering Mistaken Selections\end{minipage} \\ \hline
Detergent 1 & 564 & 14 & 14 & 2\\ \hline
Detergent 2 & 609 & 9 & 50 & 0\\ \hline
Shaving Gel & 910 & 12 & 11 & 1\\ \hline
\end{tabular}
\caption{sEMG Experiment 1 Results}
\label{tab:semg_table_1}
\end{table*}

\subsubsection{Discussion}
These results, while promising from the perspective that a single experiment participant was able to understand and utilize the system fairly quickly, demonstrated a number of shortcomings with our system. First, the user's control over the sEMG device was not very accurate, yielding many false initial selections that had to be timed out. This may be because the user was trained to use the PA muscle, which is smaller and has more variable performance. Time restraints did not allow for extensive training of the subject, and so when switching from training to task, the additional cognitive load appears to have degraded performance. A second problem was that the online reachability tester is fairly slow using the CBiRRT planner, and thus new available grasps appeared slowly. This caused relatively few reasonable grasps to be available, and so the user had more trouble because he had to iterate through more grasps which were not reflective of their intent while looking for a reasonable one. While indicating poor regions for grasping by shading the display hand was somewhat effective at helping the user avoid long waits in regions that were doomed to failure, it was not sufficient near border regions where grasps were possible but unlikely because of occlusions. 

\section*{System 4: A practical assistive grasping platform}
\setcounter{subsection}{0}
\renewcommand*{\theHsection}{chX.\the\value{section}}
Our initial results showed enough efficacy of this system that we developed a second prototype using a smaller, lower weight robotic arm, the Kinova Mico. Our initial prototype used an industrial arm, which is extremely accurate and has a large workspace, but is too heavy and expensive to be part of an assistive robotic setup. Additionally, this large, high precision arm does not reflect the performance characteristics of an arm which is affordable and practical for a robotic wheel chair. The Kinova Mico arm is more suitable for mounting on a wheel chair. 
We also sought feedback from our colleagues at the Columbia Medical Center who worked with this same sEMG device in stroke patients. Their advice was that our user interface needed further streamlining. We also sought to resolve the online reachability checking issue by integrating a faster planner. 

\subsection{Adaptations For the Mico Manipulator}
The Kinova Mico arm is a six DOF arm with a two finger gripper. The fingers each have two joints coupled to a passive under-actuation mechanism that enables both enveloping grasps of convex cross sections of objects and fingertip grasps. These fingers are made of a hard plastic which has relatively little friction, which implies that the fingers of the hand must be well aligned to the surface of the object to achieve a stable grasp. 

The transmission of the under-actuation mechanism of the hand is designed such that the fingertips remain at roughly same angle relative to the palm through most of the range of the finger's motion, similar to the motion of a parallel jaw gripper. For hands of this type, we can trivially estimate the contact point of the grasps without performing the kinematic simulation of closing the hand in GraspIt!, which is the most computationally expensive aspect of grasp analysis. In this work, we applied 10x multiplier to the quality measure of grasps whose estimated contacts aligned to within $3^{\circ}$ of the normal to the nearest surface. This was sufficient to generate only well aligned, reasonable grasp candidates. 

%to do this is to add a number of parallel virtual contacts to the grasping surfaces of the finger, as shown in Fig. \ref{fig:mico_finger_virtuals}. Additionally,

\subsection{Improved Online Reachability Checking}
Given our previous insight that the online reachability testing is a bottleneck for the online grasp refinement, we wanted to explore a different options for online reachability checking. This motivated us replace the OpenRave trajectory planner with with the MoveIt! planning environment (see \cite{moveit}), which interfaces with a large number of planners in the OMPL planning library \cite{ompl}.

The OMPL planners have different strategies with different performance properties. In order to investigate which one is appropriate to grasping in the cluttered scenes with the Mico Arm, we captured 10 scenes similar to that in Figure \ref{fig:ui-4-total-a} and ran the online reachability checker on the set of default grasps for each of the objects in the scene. Since many of the grasps in the online planner tend to be very similar, we perturbed the grasps by a +/- 0.005 m in each direction, testing 60 grasps for each of three objects for each scene. 

The online reachability check is the final stage of filtering before grasps are presented to the user. The sampling nature of the planner implies that there will be a great deal of temporal correlation between grasp requests. In order to take advantage of this correlation, we implemented a plan caching scheme which stores the start and end point of the arm trajectory in a nearest neighbors lookup tree. When planning a new trajectory for online analysis, we first attempt to plan from the end of the nearest endpoint. If that fails, we retry from the original starting position. If this second attempt succeeds, the planned path is inserted into the cache. For the actual arm motion, we retry the planning until it succeeds from the original starting location, so long as a valid cached plan exists. This is because smoothing such plans to remove the excess waypoints introduced by the initial segment from the cached plan is still an open area of research that we did not wish to address in this work.  

Because the trajectory planners are stochastic, their performance is highly task specific and sensitive to parameters such as minimum segment length and allowed planning time. We did a parameter sweep of the allowed trajectory segment length from 0.01 to 0.1 in steps of 0.01 with allowed planning times up to 20 seconds. Two of the Probablistic Road Map (PRM) planners (\cite{PRM}) planners performed the best using the caching scheme, succeeding in 43\% of the grasps, and the vanilla PRM implementation had the fastest planning for the caching version of the planner, with an average planning time of 5.5 seconds for arm motions in which the caching fails, and 0.1 seconds when the cache succeeds. 

The single-query bidirectional variant of the PRM planner (SBL) produced plans that seemed smoother in the region near the object. Planning grasps in a cluttered scene is state space in which there is a very narrow valid region near the goal state, and so one might expect a bidirectional planner to find a more optimal path out of that region because it will spend more resources directly on that part of the problem. 

However, we found empirically, when the caching scheme failed to find a reasonable neighbor, the SBL planner's success rate dropped to 30\%, whereas the PRM planner's success rate remained the same. This led to slight lag in performance as the cache was populated. So, for the online reachability verification, we used the PRM planner with a segment length of 0.05, while for producing the actual grasp on the robot we used the SBL planner. These changes removed the online reachability checking as a bottleneck for the online grasp refinement phase of the pipeline. \footnote{Although MoveIt! includes a benchmarking suite for determining the optimal parameters for a set of problems, it cannot be used with MoveIt!'s pick and place grasping pipeline, which handles the approach and lift phases of the path planning, or with robots that have some joints with continuous joint ranges. As such, we implemented our own ad-hoc optimization script.}

\subsection{Further UI Improvements}
In our previous systems, the two 'continuous' inputs which shifted the hand around the object were not part of the pipeline guide display which showed the user which phase they were in and what their inputs would do. In every phase of the pipeline, they always did the same thing. However, in this system, the purpose of these inputs can also change in each phase. In addition, our previous test user indicated that there should be a clearer differentiation between the augmented reality region containing the grasp planning scene and the rest of the UI and fewer grasp options presented during the parts of the pipeline where they are not needed.

In this system, the UI window is adaptive, providing more visual cues to the user as to what their goal is in each particular phase of the pipeline. The grasp previews are integrated with the pipeline guide display, and the pipeline guide areas also function as GUI buttons for the experimenter to use when familiarizing the subject with the UI. Each of the targets now has a corresponding color coded button. These new UI elements are shown in Figure \ref{fig:ui-4-object-selection} and Figure \ref{fig:ui-4-total-a}. While this may seem like unnecessary complexity, the UI is more visibly different in each phase and less extraneous information is presented. This seems to help subjects keep track of what phase they are in and what its purpose is. 

\begin{figure}
\centering
\begin{subfigure}[t]{\columnwidth}
\centering
\includegraphics[height=1.7in,width=.99\columnwidth]{images_4/object_recognition_state.png}
\caption{\emph{System 4 - Object Recognition and Selection State.} The graspable objects in the seen are highlighted in red and green. Sending input 1 selects the green object as the target, input 2 cycles to the next object, and input 3 triggers the object recognition system to refresh.   The background UI area is rendered in red while the recognition is still processing.}
\label{fig:ui-4-object-selection}
\end{subfigure}
\begin{subfigure}[t]{\columnwidth}
\centering
\includegraphics[height=1.7in,width=.99\columnwidth]{images_4/matlab_ui.png}
\caption{\emph{System 4 - Grasp Refinement State.} The buttons on the right function as both guides for the result of hitting the color coded input options that will be presented to the use, as well as buttons that the user and experimenter use during the training stage. The sEMG is interface overlaid on the planning scene with the selected target highlighted in green.}
\label{fig:ui-4-total-a}
\end{subfigure}
\end{figure}

\subsection*{System 4 Pipeline}
The updated pipeline is slightly shorter and makes more varied use of inputs 3 and 4. 

\emph{Object Recognition and Selection:} This phase combines the first two phases of \emph{System 3}. To select an object as a target, the user sends input 2. To cycle to the next object in the recognized object list the user sends input 3, which will continuously iterate through the grasps until the user leaves the rest area. To rerun the object recognition system the user sends input 1. While the recognition system is still running, the whole screen is highlighted in red and it is not possible to proceed to the next phase until the recognition finishes. 

\emph{Initial Review:} As in the \emph{System 3}, the user is presented with a list of preplanned grasps from a precomputed database. The UI presented is shown in Figure \ref{fig:ui-4-total-a}, in which the currently selected grasp is shown in the window of the top of the guide area, with the color of the background again indicating the results of the online reachability checker. The next grasp is shown in the bottom of the window. Input 1 begins the online refinement stage, input 2 skips to the Final Grasp Review phase. Input 3 will iterate through the available grasps, whereas input 4 will return to the Object Recognition and Selection Phase.

\emph{Grasp Refinement:} This phase is similar to \emph{System 3}, but with one fewer grasp displayed more prominently. The first grasp shown in the top of the window and the next grasp shown on the bottom. Input 1 proceeds to the Final Grasp Review phase, input 2 aligns the hand to the next grasp and brings it up to the top window. Inputs 3 and 4 rotate the hand around the object as previously described. 

\emph{Final Grasp Review:} As in the previous phase, this phase has been adapted to have only two grasps, the top showing the current selection and the bottom showing the next selection. Input 1 proceeds to the Grasp Choice Confirmation phase, input 2 aligns the hand to the next grasp and brings it up to the top window. Inputs 3 and 4 rotate the hand around the object as previously described. 

\emph{Grasp Choice Confirmation:} This phase is similar to \emph{System 3}, but with only the selected grasp shown in the grasp preview window. The user sends input 1 to go back to the Grasp Refinement phase and input 2 to send the grasp for execution on the robot.

\subsection{Validation}
To validate these design decisions, we tested our pipeline on 5 healthy subjects, 2 male and 3 female, ages 22-30. All testing was approved by the Institutional Review Board of the Columbia University under Protocol AAAJ6951. To simplify testing of the UI, we did not attempt to train the subjects on the two dimensional version of the user interface. Instead, the subjects were given a similar user interface, but the cursor is constrained to move towards the target representing the 'selected' input, which is outlined in green as shown in Figure \ref{fig:ui-4-total-a}. In order to switch which target is currently 'selected', the user leaves the rest area and returns to it without hitting a target. This cycles the 'selected' target forward by one. This change allowed us to focus on testing improvements the user interface and grasp planning pipeline without needing the more extensive training necessary to train a subject to achieve full 2D control over the cursor. 

\subsubsection{sEMG Device Setup}

\begin{figure}
\centering
\includegraphics[width=.49\textwidth]{images_4/semg_clutter_grasp.png}
\hspace{1mm}
\includegraphics[width=.49\textwidth,trim={30cm 40cm 20cm 20cm},clip=true]{user_semg_2.jpg}
\caption{\emph{Top:} A typical grasp of the shampoo bottle from the side in the cluttered scene. Note that the hand is just able to fit between the other objects to grasp the desired target. Note that the ability to plan this grasp in such a restricted environment is an indication that this system is very successful at handling the cluttered scene. \emph{Bottom:} The sEMG system electrodes. In these experiments, we placed the electrodes behind the ear of the subject to measure contractions of the PA muscle. We stabilize the electrodes by wrapping the head of the electrodes in Silly Putty silicone putty.}

\label{fig:silly-putty}
\end{figure}

In these experiments, we placed the sEMG device behind the ear of the subject to measure contractions of the PA muscle. In order to stabilize the device and reduce noise due to motion of the wires, we stabilize the electrodes by wrapping the head of the electrodes in Silly Putty\textsuperscript{TM} silicone putty, as shown in Figure \ref{fig:silly-putty}. We find the correct placement of the device by asking the subject to clench their jaw gently and raise their eyebrows. We place the electrodes where we find a large response to eyebrow raises and  little response to jaw motion.

\subsubsection{Training}
\emph{sEMG Device:} Each subject was trained on the sEMG user interface without the grasp planning system. In the training system, the user is given a \emph{desired target} highlighted in red which is randomly selected at the beginning of each trial. The user is then instructed to cycle the \emph{selected target} until it overlaps with the \emph{desired target}, which is then shown in gold. The subject was asked to perform sets of 30 trial blocks until they successfully completed at least 29 of the 30 attempts. This took at most 2 blocks of trials for any subject, with subjects who already had some ability to move their ears frequently succeeding in their first block. 

\emph{Grasp Planning Interface:} To familiarize the subject with the grasp planning system, we manually showed the subject three examples of grasping objects, once short circuiting the online planning and twice allowing the online refinement to run. Then we allowed the subject to guide the pipeline themselves five times, twice without the online planner and three times with it. Then we repeated the training allowing the subject to guide the planner to pick up the large detergent bottle five times in whatever direction they chose using the UI through the on screen button interface. 

\subsubsection{Task}
We placed the objects on the table in proximity to one another as shown in Figure \ref{fig:ui-4-total-a}. We asked the subject to grasp each object three times, the first time from any direction they deemed reasonable, once from the side, and once from above. For each object, the placement of the objects and grasps in the database were such that either the side or top grasp required the online grasp refinement. Since the workspace of the Mico arm is not very large, it is easy to find such object positions. 

\subsubsection{Results}
\label{sec:semg_results}
The results of the experiment for 5 subjects are tabulated in Tab. \ref{tab:results_3}. On average, the subjects were successful in grasping 82\% of the objects within 92 seconds of the first time their cursor left the rest area. With respect to speed,  results are comparable, and indeed somewhat better than the amount of time it took subjects to grasp objects with the Emotiv Epoc, even though the subject may have to iterate over the possible options before selecting them. Subjects 2 and 3 were the best able to control the cursor, having previously been able to move their ears already, and also performed the best in these experiments. These results indicate that the underlying planning system is providing options that the less capable subjects are not exploring because they are having more difficult with the UI. 

For the shampoo bottle, there are relatively fewer grasps that can succeed as compared to the rotationally symmetric shaving gel bottle and the taller, sloping detergent bottle. The only feasible grasps from the side for the shampoo bottle are directly from the side, aligned with the wide axis of the bottle, as demonstrated in Figure \ref{fig:silly-putty}. This narrow feasible region and the potential for many collisions with the other objects in the scene during the reaching motion to this region makes this a particularly difficult grasp, especially when the clearance around the grasp is as tight as it is in Figure \ref{fig:silly-putty}. Without the partial plan caching implemented in the online trajectory planner, planning grasps to this region using stochastic, sampling based planners is extremely unreliable. With the caching scheme, this grasp was successfully found 100\% of the time, although the planning time is somewhat longer than the other grasp tasks.  

% \singlespace
% \begin{table}[t]
% \raggedleft
% \begin{minipage}[t]{0.4\columnwidth}
% \begin{tabular}[t!]{ | c c c c | }
% \hline
% Grasp & Subject & Success & Time \\ \hline \hline
% \multirow{6}{*}{\begin{minipage}[t]{0.2\columnwidth}Detergent Bottle Top\end{minipage}} & 1 & Yes & 75 \\ 
% & 2 & Yes & 53 \\ 
% & 3 & No & 45 \\
% & 4 & No & 122 \\
% & 5 & Yes & 135 \\ 
% & Mean & 60\% & 86\\\hline
% \multirow{6}{*}{\begin{minipage}[t]{0.2\columnwidth}Detergent Bottle Side\end{minipage}} & 1 & No & 66 \\ 
% & 2 & Yes & 40 \\ 
% & 3 & Yes & 52 \\
% & 4 & Yes & 80 \\
% & 5 & Yes & 85 \\ 
% & Mean & 80\% & 64\\\hline
% \multirow{6}{*}{\begin{minipage}[t]{0.2\columnwidth}Detergent Bottle Open Choice\end{minipage}} & 1 & Yes & 50 \\ 
% & 2 & Yes & 57 \\ 
% & 3 & Yes & 53 \\
% & 4 & Yes & 135 \\
% & 5 & Yes & 128 \\ 
% & Mean & 100\% & 85\\\hline
% \multirow{6}{*}{\begin{minipage}[t]{0.2\columnwidth}Shampoo Bottle Top\end{minipage}} & 1 & Yes & 151 \\ 
% & 2 & Yes & 72 \\ 
% & 3 & Yes & 60\\
% & 4 & No & 126 \\
% & 5 & No & 104 \\ 
% & Mean & 60\% & 102\\\hline
% \multirow{6}{*}{\begin{minipage}[t]{0.2\columnwidth}Shampoo Bottle Side\end{minipage}} & 1 & Yes & 134 \\
% &2 & Yes & 95 \\
% &3 & Yes & 132 \\
% &4 & Yes & 164 \\
% &5 & Yes & 143 \\ 
% & Mean & 100\% & 133\\\hline
% \end{tabular}
% \end{minipage}
% \hspace{.01\columnwidth}
% \raggedright
% \begin{minipage}[!t]{.4\columnwidth}
% \begin{tabular}[t!]{ | c c c c | }
% \hline
% Grasp & Subject & Success & Time \\ \hline \hline
% \multirow{6}{*}{\begin{minipage}[t]{0.2\columnwidth}Shampoo Bottle Open Choice\end{minipage}} & 1 & Yes & 93 \\
% &2 & Yes & 121 \\
% &3 & Yes & 63 \\
% &4 & Yes & 95 \\
% &5 & Yes & 117 \\ 
% & Mean & 100\% & 98\\\hline 
% \multirow{6}{*}{\begin{minipage}[t]{0.2\columnwidth}Shaving Gel Top\end{minipage}} & 1 & No & 83 \\ 
% & 2 & No & 123 \\ 
% & 3 & Yes & 112\\
% & 4 & No & 139 \\
% & 5 & Yes & 97 \\ 
% & Mean & 60\% & 111\\\hline 
% \multirow{6}{*}{\begin{minipage}[t]{0.2\columnwidth}Shaving Gel Side\end{minipage}} & 1 & Yes & 65 \\
% &2 & Yes & 52 \\
% &3 & Yes & 57 \\
% &4 & Yes & 88 \\
% &5 & Yes & 92 \\ 
% & Mean & 100\% & 71\\\hline 
% \multirow{6}{*}{\begin{minipage}[t]{0.2\columnwidth}Shaving Gel Open Choice\end{minipage}} & 1 & No & 73 \\
% &2 & Yes & 59 \\
% &3 & Yes & 76 \\
% &4 & Yes & 81 \\
% &5 & Yes & 85 \\ 
% & Mean & 80\% & 75\\\hline 
% \multirow{6}{*}{\begin{minipage}[t]{0.2\columnwidth}Average Performance\end{minipage}} & 1 & 66\% & 87 \\
% &2 & 88\% & 75 \\
% &3 & 88\% & 72 \\
% &4 & 77\% & 114 \\
% &5 & 88\% &  109\\ 
% & Mean & 82\% & 92\\\hline 
% \end{tabular}
% \end{minipage}
% \caption{Results from Experiment 3. On average, the subjects were successful in grasping 82\% of the objects within 92 seconds of the first time their cursor left rest area.}
% \label{tab:results_3}
% \end{table}
% \doublespace

%\singlespace
\begin{table*}[ht!]
\raggedleft
\begin{minipage}[t]{0.5\textwidth}
\begin{tabular}[t!]{ | c c c c | }
\hline
Grasp & Subject & Success & Time \\ \hline \hline
\multirow{6}{*}{\begin{minipage}[t]{0.2\columnwidth}Detergent Bottle Top\end{minipage}} & 1 & Yes & 75 \\ 
& 2 & Yes & 53 \\ 
& 3 & No & 45 \\
& 4 & No & 122 \\
& 5 & Yes & 135 \\ 
& Mean & 60\% & 86\\\hline
\multirow{6}{*}{\begin{minipage}[t]{0.2\columnwidth}Detergent Bottle Side\end{minipage}} & 1 & No & 66 \\ 
& 2 & Yes & 40 \\ 
& 3 & Yes & 52 \\
& 4 & Yes & 80 \\
& 5 & Yes & 85 \\ 
& Mean & 80\% & 64\\\hline
\multirow{6}{*}{\begin{minipage}[t]{0.2\columnwidth}Detergent Bottle Open Choice\end{minipage}} & 1 & Yes & 50 \\ 
& 2 & Yes & 57 \\ 
& 3 & Yes & 53 \\
& 4 & Yes & 135 \\
& 5 & Yes & 128 \\ 
& Mean & 100\% & 85\\\hline
\multirow{6}{*}{\begin{minipage}[t]{0.2\columnwidth}Shampoo Bottle Top\end{minipage}} & 1 & Yes & 151 \\ 
& 2 & Yes & 72 \\ 
& 3 & Yes & 60\\
& 4 & No & 126 \\
& 5 & No & 104 \\ 
& Mean & 60\% & 102\\\hline
\multirow{6}{*}{\begin{minipage}[t]{0.2\columnwidth}Shampoo Bottle Side\end{minipage}} & 1 & Yes & 134 \\
&2 & Yes & 95 \\
&3 & Yes & 132 \\
&4 & Yes & 164 \\
&5 & Yes & 143 \\ 
& Mean & 100\% & 133\\\hline
\end{tabular}
\end{minipage}
\raggedleft
\begin{minipage}[!t]{.4\textwidth}
\begin{tabular}[t!]{ | c c c | }
\hline
Grasp & Success & Time \\ \hline \hline
\multirow{6}{*}{\begin{minipage}[t]{0.2\columnwidth}Shampoo Bottle Open Choice\end{minipage}}& Yes & 93 \\
& Yes & 121 \\
& Yes & 63 \\
& Yes & 95 \\
& Yes & 117 \\ 
& 100\% & 98\\\hline 
\multirow{6}{*}{\begin{minipage}[t]{0.2\columnwidth}Shaving Gel Top\end{minipage}} & No & 83 \\ 
& No & 123 \\ 
& Yes & 112\\
& No & 139 \\
& Yes & 97 \\ 
&  60\% & 111\\\hline 
\multirow{6}{*}{\begin{minipage}[t]{0.2\columnwidth}Shaving Gel Side\end{minipage}} & Yes & 65 \\
& Yes & 52 \\
& Yes & 57 \\
& Yes & 88 \\
& Yes & 92 \\ 
& 100\% & 71\\\hline 
\multirow{6}{*}{\begin{minipage}[t]{0.2\columnwidth}Shaving Gel Open Choice\end{minipage}} & No & 73 \\
& Yes & 59 \\
& Yes & 76 \\
& Yes & 81 \\
& Yes & 85 \\ 
& 80\% & 75\\\hline 
\multirow{6}{*}{\begin{minipage}[t]{0.2\columnwidth}Average Performance\end{minipage}} & 66\% & 87 \\
& 88\% & 75 \\
& 88\% & 72 \\
& 77\% & 114 \\
& 88\% &  109\\ 
& 82\% & 92\\\hline 
\end{tabular}
\end{minipage}
\caption{Results from Experiment 3. On average, the subjects were successful in grasping 82\% of the objects within 92 seconds of the first time their cursor left rest area.}
\label{tab:results_3}
\end{table*}
%\doublespace

\section{Conclusions}
In this paper, we've discussed the many details involved in building a full assistive grasping system around an online grasp planner. The key challenge was to find the right balance of complexity and usability, particularly with respect to the design of the visual aspects of the interface. A clear user interface is the key to allow a non-expert user to apply their intuition to the grasping problem and provide the added value that makes the system work well in spite of sensor noise and any shortcomings in the heuristics applied by the automated parts of the system. Careful development of this platform has allowed us to produce an extremely capable system around components whose cost and complexity is not prohibitive. 

Through this work with the UC-Davis sEMG device, we have pushed the boundaries of what can be accomplished with a minimally invasive, facial muscle driven input. First, we extended our basic system design to a more complex environment with multiple objects in close proximity to one another. This involved augmenting the user interface with additional phases to select the desired object, adding an online reachability tester, and producing a new UI with a dedicated interface including a cleaner UI with an integrated sEMG driven option selection overlay. After initial validation of the interface on an impaired user, we developed a series of improvements to the user interface, the online grasp planning, and online reachability filter to address the most challenging issues that caused our initial user to take up to eight minutes to make a single grasp selection. We developed a novel control paradigm for testing these changes without changing the visual interface which allowed us to validate the updated system on naive users without the extensive training necessary to train an individual to develop full 2-D control. 

This study serves as a pilot to validate the design choices of the system on a path towards more experiments with impaired users. Even though this paradigm requires the user to make up to four motions for selections which had previously required one, we observed that users took on average 1/8th the time to make grasp selections in the latest version of the sEMG driven planner. We did not explicitly measure how long the users spent in each stage of the pipeline, but one of the most costly phases was observed to be the grasp refinement stage, when it was used. In order to improve performance in this stage, we would have to improve the performance of the collision detection system, which is the dominant cost of the simulated annealing driven grasp refinement. Overall, the majority of the failures to grasp an object were caused by the difficulty of grasping cylinders along the major axis with a gripper, represented by grasping the detergent bottle or shaving gel bottle from above. In these grasps, squeezing the gripper can easily cause the object to be ejected. Subjects cannot seem to learn to expect this behavior without having experienced it a number of times, as they do not have a good sense of the friction properties of the gripper. To improve this behavior, we would have to implement a more complex feedback controller during the grasping process. It is also likely that with greater experience, the subjects would have been more familiar with the kind of grasps that cause ejection. 

This work demonstrates one the first EMG driven grasping systems that we know of that allows a user to grasp an object in a somewhat cluttered scene, or integrates user intent with the intermediate level of control we have proposed. The sEMG device itself is very minimalistic, and could itself be embedded in the frame of a pair of glasses, which makes this device a real candidate for evolving to a consumer level product. Future work for this paradigm will be to refine the training paradigm to make learning 2D control over the device easier, exploring new control paradigms, and extending the user interface to control the locomotion of a motorized wheel chair or mobile manipulator assistive platform. We have also begun some exploration towards more complex grasp quality measures that integrate more complex artificial intelligence techniques such as ``deep learning'' which may have less reliance on the accuracy of the object recognition system. 


%%%%Named reference model and we can use the same as numbered reference
%%%%giving option by using natbib package


\bibliographystyle{named}
\bibliography{references/refs,references/references2,references/bci_refs2,references/grasp_quality}




%%%%%%%%DOCUMENT END HERE
\end{document}




%%%%%%%%%%%%%%%%%%%%%%%%%%%%%%%%%%%%%%%%%%
%%%%%%%%%%%%%%%%%FOR BIB DATABASE%%%%%%%%%



